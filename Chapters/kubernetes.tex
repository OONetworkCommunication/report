%************************************************
\chapter{Kubernetes}\label{ch:kubernetes}
%************************************************
Topic 3: Write a brief introduction to the Kubernetes container cluster manager. 
Design, develop, test, and document a functional proof-of-concept prototype using the Kubernetes container cluster manager. 

This topic builds on top of the previous two topics 1 and 2 and should be able to:\\
- deploy all containerized services in Kubernetes on the Raspberry Pi cluster \\
- scale pods in Kubernetes \\
- update pods with rolling-update \\

\textbf{What is Kubernetes?}\\
Kubernetes is an open-source container management system developed by Google.
Kubernetes is used for automating management of containers such as scaling, loadbalancing and scheduling containers between nodes in a cluster. Containers are run inside something called a pod.
A pod is a group of containers that are scheduled onto the same host.
Pods are used for gathering containers that are tightly coupled to each other under the same host such communication between containers becomes easier. Every pod has it's own ip address. Kubernetes schedules, deploy and scale pods such they among other ensure that the load is balanced in the cluster. 

The containers in the nodes are replicated, so if one node fails Kubernetes will create a new pod on another node and run the container. 
If there suddenly comes a lot of load on a node it will make extra \textbf{replicaset} of the pods so that the load is more balanced. 
