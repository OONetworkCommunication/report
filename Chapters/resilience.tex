%************************************************
\chapter{Resilience}\label{ch:resilience}
%************************************************
Topic 4: Write a brief introduction to Resilience in microservice architectures and to how Gatling load testing framework can be used for testing. Run load testing scenarios against the handed out containers and describe the effects of using patterns.
- run different load testing scenarios handed out
- hand-in generated reports and key metrics on BlackBoard 
- hand in exercise report
- run load-testing against your project architecture to see the effects

The word resilience mean "The capacity to recover quickly from difficulties". there are many definitions of the resilience, one of the definitions is taken from the book "The Resilience of Networked Infrastructure Systems". Fiskel (2003) defines a resilient system as a system that has the ability to return to a stable equilibrium state after a perturbation. 

There are different categories of potential disruptions to system that create the need to implement the resilience in systems. In the figure \ref{ch:resilience} below the sources of disruptions is showed. 

\begin{figure}[bth]
	\includegraphics[width=0.7\linewidth]{gfx/resilience}
	\caption[routingtable]{Sources of disruptions} \label{fig:resilience}
\end{figure}   

\begin{itemize}
	\item Human Factor
	\begin{itemize}
		\item One of the common disruption is caused by human factors. The disruption in this case could be external attacks on the system, or just so simply as human using the system.  
	\end{itemize}
	\item Natural Factors
	\begin{itemize}
		\item A infrequent influence on damaging the system caused by natural disaster, such as floods and hurricane. 
	\end{itemize}
	\item Organizational factors
	\begin{itemize}
		\item Another difficulty to be considered about disruption the system is organizational factors, such as the worker strikes.   
	\end{itemize}
	\item Technical Factors
	\begin{itemize}
		\item Technical difficulty need also to be considered in the system. It is common that system component fails, and need to be either replaced or to be repaired.   
	\end{itemize}	
\end{itemize}

All the factors described above need to be considered in the system to avoid disruption. 




