%************************************************
\chapter{Docker}\label{ch:docker}
%************************************************
Topic 2: Write a brief introduction to Docker containers.  Design, develop, test, and document a functional proof-of-concept prototype using Docker containers. This builds on top of topic 1 and should be able to
- build docker containers of each service in topic 1
- run docker containers locally and on a Raspberry Pi
- deploy images to Docker Hub


Docker containers allows packaging of software application with all its dependencies and libraries. It makes it easier to create, deploy and run the applications. The main purpose of Docker is to ship the whole application as a single package. 
By doing that the Docker ensures that the application will run on any other Linux machine regardless of any customized settings the particular machine might have.
Docker has resemblance to virtual machines in a way, but Docker allows applications to use the same Linux kernel as the system that they’re running on. By using Docker one can bypass creating virtual machine for every application. 
An algorithm can be run in docker using any Linux compatible language such as C, Python, Matlab etc. and be compiled using any Linux compatible libraries without causing version or library conflicts with other algorithms.



A docker container can be run at any time with the confidence of that the docker container’s computational environment will be identical. The docker projects remains backwards compatible. 


There 2 significantly different ways of building Docker containers:

\begin{}
\item interactively
\item Dockerfile
\end[{} 

Building a container interactively makes it possible to install libraries and configure the environment from a shell just like in a typical Linux environment. The modifications can be saved via commits identical to Git, it’s possible to track changes and see the status of modification.  

On the other hand Dockerfile builds a container entirely through commands. A Dockerfile identifies a source container to start from (typically a basic Linux installation), then records a series of commands to install libraries and configure the environment. Dockerfiles can also load other Dockerfiles allowing for environment layering and concise organization of various software development projects. Dockerfiles can be versioned using a standard source control versioning tool like Git, allowing revision tracking and archiving of the computational environment.
