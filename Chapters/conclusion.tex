%************************************************
\chapter{Conclusion}\label{ch:conclusion}
%************************************************

The implementation of cloud using Microservices along with Docker and Kubernetes made sense to use instead of using monolithic service. The purpose of this project was to get a taste of how the Microservices work within the domain of Cloud Computing. 

In this project the implementation of Config Server for the spring services made the overall services dynamics better and made it easier to modify the services. 

Docker was there to pack easy thing and by using Docker the static environment was achieved, which makes the services quite flexible because they can run within all OS. 

Last but not least the deployment of Kubernetes on top of Spring boot and Docker made it easier to manage the resources and automatically control the load on individual nodes. 

The big conclusion is that the these 3 technologies together makes the Cloud Computing easier because the concept of having Monolithic services is avoided. 

By using these 3 technologies the Cloud Computing system could contain thousands of services. The big challenge is to managing the requests between the services. When there are so many services, the services needs to communicate with each other and they do that by having REST API and there are dozens thousands of requests sent easy single time there is some computation. 

There is utilized more bandwidth when using Microservices then Monolithic services. If the bandwidth is small then there can occur the phonomina of latency which can destroy overall performance of the system. 