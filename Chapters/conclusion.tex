%************************************************
\chapter{Conclusion}\label{ch:conclusion}
%************************************************

This, then is the grand summary of what you have accomplished.  You
may well imagine that many readers will read your Introduction, and
then skip to the Conclusion, and if, and only if, those two parts are
interesting, might be tempted to read the rest. A consequence is that
you should ensure that the reader will gain a good overall
understanding of what you have done by reading only the conclusion.
Thus, this is a place to summarise all that has gone before, before
finally concluding on the results of your experiments and the validity
of your hypotheses. It is also important to ensure that the
Introduction (which in all likelihood was written first) still aligns
closely with the conclusions reached.

If you so desire, this is also where you might add a section on Future
Work, where you point in the directions that should be followed to
complete the work you have already accomplished.